\section{Machine Model with Devices}
\label{sec:model}

In this section, we present our machine model, which is based on the Intel x86
architecture. We start from the {\it LAsm} machine model, and extend it
to model devices and interrupts.

Our devices are modeled as finite state transition systems interacting with the
CPU and the external environments. Each read/write (input/output) operation
initiated from the CPU triggers an atomic big-step transition in the
corresponding device.  Device transitions (i.e., $\textsf{trans}$ in
Fig.~\ref{fig:driver}) are affected by two types of interactions, one by the CPU
and another by external events.

\paragraph{Device Transitions caused by the CPU} The CPU may trigger a
device transition through I/O instructions or memory-mapped I/O
operations. These operations can be categorized into the following
two actions:

\begin{definition}[CPU Operation on a Device]
\[
\begin{array}{rll}
\mathcal{O}::= & \textsf{input} ~  n  &~~~~~~\vartriangleright \text{Read value from the register at address $n$} \\
| & \textsf{output}  ~ n ~ v &~~~~~~\vartriangleright \text{Write value $v$ to the register at address $n$}
\end{array}
\]
\end{definition}

For every device, we define an atomic transition function
$\delta^{\textsf{CPU}}$, which takes the current device state $s$ and
a CPU operation $o$, and returns the new state $s'$. Note that
$\delta^{\textsf{CPU}}$ is not a CPU transition, instead, it is strictly a
{\em device transition} triggered by a CPU I/O operation.

\paragraph{Device Transitions caused by External Events}
Device transitions can also be caused by events from the external
environment, such as the keyboard or network, with specific
transitions depending on the kind of event. When modeling these
external events, we take a minimalistic approach: though the devices
can receive all kinds of different external events, we only model
those that change the observable behavior of the device.  Thus, the
events do not map one to one to the transitions in the device hardware but
rather to the CPU observations on the hardware. We model the device
interfaces, not the device internals. The device interface contains
all the information that a programmer can know about its states.
% or interaction with peripheral devices is via the CPU intermediary.
Some example events are:

\begin{definition}[Device External Events] \label{def:eevent}
\[
\begin{array}{rll}
E ::= &  &  \\
 \multicolumn{3}{l}{\vartriangleright \textit{UART device}} \\
 | & \textsf{Recv} ~ (s: \textsf{list} ~ \mathsf{char}) &~~~~~~\vartriangleright \textit{UART receives string s} \\
 | & \textsf{NoSendingCompAck} &~~~~~~\vartriangleright \textit{Sending is not complete} \\
 | & \textsf{SendingCompAck} &~~~~~~\vartriangleright \textit{UART completes the sending} \\
 \multicolumn{3}{l}{\vartriangleright \textit{Keyboard device}} \\
 | & \textsf{KeyPressed} ~ (c: \mathbb{Z}) &~~~~~~\vartriangleright \textit{A specific key is pressed} \\
 | & \textsf{KeyReleased} ~ (c: \mathbb{Z}) &~~~~~~\vartriangleright \textit{A specific key is released} \\
 \multicolumn{3}{l}{\cdots} \\
\end{array}
\]
\end{definition}

External events are unpredictable, as their causes are not controlled by the OS.
We determinize the behavior of each device by parametrizing it with
the set of all possible list of events $\ell^{env}$ that will be processed
sequentially when the CPU performs I/O operations
on this device.
The atomic transition function
$\delta^{\textsf{env}}$ takes an external event $e$ as input and changes the device states
accordingly.

Note that events, even within a single device, can commute. For
example, a serial port serves two roles: to receive user input
and to send program output.  Accordingly, among the events a serial
device can receive are one for the reception of a new input string,
and one signaling that some past output operation has been
completed. Consider a function that first writes to a serial port,
then waits until the write operation is completed by repeatedly
reading some relevant status register. During one of these reads the
user might send new input to the serial port. It would be reasonable
for the device to observe the corresponding {\it \textsf{Recv}} event
during one of the register reads, but doing so would make verifying
the write function unnecessarily complex; not only would a function
need to handle its own logic, but it would also need to handle any
other state transition the device could undergo, even if the result
were not observable in the current function.

\ignore{
\hao{\sout{
One way to mitigate this complexity would be to add invariants to the sequence of
observed events $\ell^{env}$, requiring that some events are in a predefined pattern appears at a point
when the device driver wishes to handle it, but, taken too far, this would make
$\ell^{env}$ contain only a subset of possible input-sequences, thus making the
model unrealistic. Instead, we add rules for how devices observe events.
}} \hao{it is not the correct way}
}

To address this verification challenge, each device keeps a set of
local logs $\vec{\ell} = \{\ell_1, ... \ell_k\}$, each of which is a strict prefix
of $\ell^{env}$.  
\footnote{We have chosen the prefix form over the subset to allow us
determine more easily where the current execution is at on the global
event list.}
The serial device from the example above could
contain two local logs, one for input and one for output.  Then when
$\delta^{\textsf{env}}$ receives an event that does not correspond to
the currently processed action, the event can simply be skipped.
\ignore{
Since the events in separate local logs commute,
when $\delta^{\textsf{env}}$ receives an event that belongs in a local log other than the one
currently being processed, the event can simply be skipped.
}
When a later action observes a part of the device state which is
affected by the event, that action will handle the event. In the
serial port example, we would defer handling the {\it \textsf{Recv}}
event until some process reads from the serial port.

Every raw device provides two I/O primitives: $\texttt{read} ~ n$ and
$\texttt{write} ~ n ~ v$. The $\texttt{read}$ primitive first updates
the device state based on the environmental device transition
$\delta^{\textsf{env}}$ with the next relevant external event in
$\ell^{env}$, then returns a value from the new state, and finally
does the transition $\delta^{\textsf{CPU}}$ triggered by this read
action. The $\texttt{write}$ primitive first triggers the transition
$\delta^{\textsf{env}}$ to update the device state based on the next
relevant external event, then performs the transition
$\delta^{\textsf{CPU}}$ initiated by this write operation.

% In this paper, we use the standard list notations of \textsf{nil},
% \textsf{cons}, \textsf{hd}, and \textsf{tl}.
In the following, the function $\textsf{next}(\ell^{env}, \ell_i)$ finds
the first relevant event $e$ in $\ell^{env}$ that has not yet been
processed with respect to the local log $\ell_i$, and returns the event
$e$ plus a new local log that is synchronized with $\ell^{env}$ up to
the event $e$.

\ignore{
, and define a new list operation:

\begin{small}
	\[ \begin{array}{lrlr}
	\ell_1 - \ell_2 & ::= & \left\{\begin{array}{ll}
	\ell_1 & \textsf{if} ~ \textsf{hd} ~ \ell_1 \neq \textsf{hd} ~ \ell_2 \\
	\textsf{tl} ~ \ell_1 - ~ \textsf{tl} ~ \ell_2 & \textsf{if} ~  \textsf{hd} ~ \ell_1 = \textsf{hd} ~ \ell_2 \\
	\end{array} \right. & \text{(Minus)} \\
	\ignore{
	
	\ell_1 \preceq \ell_2 & \equiv & \ell_1 = \textsf{nil} & (\text{Prefix}) \\
	& | & \textsf{hd} ~ \ell_1 = \textsf{hd} ~ \ell_2 \wedge \textsf{tl} ~ \ell_1 \preceq \textsf{tl} ~ \ell_2 }
	\end{array}
	\] 
\end{small}
}

Now, we define the operational semantics of the set of device primitives
formally. Let $\kappa$ be the function retrieving the value of device register
addressed by $n$, then we have:

\[
	\begin{array}{lr}
	\inferrule{
	(e, \ell'_i) = \textsf{next} (\ell^{env}, \ell_i) \\	
	s' = \delta^{\textsf{env}} (s, e) \\
	res = \kappa(n, s') \\
	s'' = \delta^{\textsf{CPU}} (s', \textsf{input} ~ n) \\
	}{
	\texttt{read} (n, s, \ell_i, \ell^{env}) \defeq (res, s'', \ell'_i) \\
	} & \text{(\texttt{read})} \\
\\
	\inferrule{
	(e, \ell'_i) = \textsf{next} (\ell^{env}, \ell_i) \\	
	s' = \delta^{\textsf{env}} (s, e) \\
	s'' = \delta^{\textsf{CPU}} (s', (\textsf{output} ~ n~ v))
	}{
	\texttt{write} (n, v, s, \ell_i, \ell^{env}) \defeq (s'', \ell'_i) \\
	} & \text{(\texttt{write})} \\
	\end{array}	
\]

\ignore{ \newman{We need to very CAREFULLY explain here that we only
    observe one ``combined'' event in every CPU operation. Then there
    will be the case where there is no external event happened since
    last operation. That's why we have the special NOEVENT event.  We
    have to be careful because this does make the event model
    unrealistic.  Furthermore, we assume that between two operations,
    we would not receive unrelated event (then it cannot be combined
    into one). We can discuss this tomorrow.}  }

Thanks to the local logs, this machine model eliminates much of the
nondeterminism that complicates reasoning about asynchronous systems.
Nonetheless, it accurately models the observable behaviors of real hardware.

Ideally, the global event list should contain a sequence of small atomic
events which gets associated with time when they get triggered. Then the framework
should have some notion of real-time clock that allows us to determine what exact
set of events should be consumed by the next transition based on the current time.
This is out of scope of the current paper. Instead, we allow our transition
to consume a single ``combined'' event (as a list of event). For example, the
``Recv'' event in the Definition \ref{def:eevent} takes a list of characters.
In addition, a device may also consume an ``empty'' event which indicates there
was no event triggered in the real world since the last time we checked $\ell^{env}$.
We parameterize our proof
in a way that it hold for all possible combinations of such combined event
list.

\ignore{
There is a mapping between each $n$ in the $\textsf{input}~n$ /
$\textsf{output}~n~v$ and the state need to be accessed. This mapping is hidden
behind the specification of function $\kappa$ and $\delta^{\textsf{CPU}}$ for
handling output events, and it essentially reflects the interfaces between the
CPU and devices. For most drivers, there are three types of methods to interact
with devices: \textit{port-mapped I/O} and \textit{memory-mapped I/O}.

For the port-mapped I/O, we introduce two primitives: $\textsf{in}~n$ and
$\textsf{out}~n~v$ to branch the port access to the raw device \textsf{read} and
\textsf{write} function. Following the ``logical CPU'' approach in
Sec.~\ref{sec:overview}, we enforce each range of ports only belongs to one
isolated machine, and that guarantees that the device cannot be messed by code
that are not in this driver. Memory-mapped I/O follows the same design. The
memory region that are used to access to one devices is also viewed as the part
of the memory belongs to that device.
}

% In the next section, we show how to extend the
% machine model to reason about interruptible code
% in the kernel.


