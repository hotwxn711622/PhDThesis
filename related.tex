\label{chapt:related}

\section{Hoare-Style Program Verification}

Hoare logic \cite{hoare69} is a popular formal system that is used to reason about correctness of programs.
Each program $C$ is annotated with the precondition $P$ and post condition $Q$ to form the Hoare triple $\{P\}~C~\{Q\}$, used to
describe the way the execution of the program $C$ changes the state of the computation.
In generic case, the Hoare triple states that for any state $S$, if $P(S)$ holds, and from state $S$, if the program $C$ executes and terminates normally
with a new state $S'$, then $Q(S')$ holds. In case we prove the total correctness, we also need to prove that if $P(S)$ holds, then
the program $C$ indeed terminates normally, instead of taking it as premise.
One important extension to the Hoare logic is the separation logic \cite{reynolds02}, which allow us to effectively reason about
programs that access and mutate shared data structures. The ``frame rule'' enables us to reason about each programs locally,
focusing only on the portion of memory being access by the program.
Verification is performed via applying the syntactical logical rules that are developed and proved to be sound.
Proof automation can be achieved via carefully designed proof tactics, or producing low level formulas that can be fed into
efficient low level solves  \cite{boogie05,dafny10}.

On the other hand, our program correctness lemma is stated in a way such that there is a forward simulation from the specification $SP$
to the implementation $C$, i.e., for the given list of arguments and initial state $S$, if $SP$ does not get stuck, and produces a new 
states $S'$ and potential return value $V$, then the program $C$ executes and terminates safely and produces the same ending state and return value.
We insist in proving the total correctness, thus need to prove that $C$ always terminate safely.
If we consider the list of conditions in $SP$ on the list of arguments and initial state that causes it to not to get stuck, as a precondition,
and the new states and return value as the post condition, our correctness lemma can be viewed very similar to the Hoare triple.
In fact, it is a stronger form of Hoare triple in the sense that we always insist on the deep specification, thus we capture all the information
we intended to know about the final state, and in fact, in most cases, the specification is a function that produces exact new abstract states upon 
execution. This does not mean that we always capture entire implementation details in the specification, as we hide all necessary details
when we abstract the concrete in-memory structure into logical abstract states, and the specification is in terms of these abstract states.
Unlike separation logic, our reasoning on isolation is completely separated from the code verification. For example, instead of reasoning
on the pointer memory access directly, we first perform data abstraction to turn the in-memory data structure into logical abstract states,
where the isolation is guaranteed by design, and provide the exclusive accesses to these abstract states via explicit list of primitives.
Next, further reasoning on isolation properties, e.g., reasoning of invariants, concurrency, interrupt, etc, are performed separated
at a pure logical level after the code verification is performed against a specification that is reasonably close to the implementation,
where we can provide strong automation support that still provide great flexibility to the users.
Since these non-trivial reasoning is done at purely logical level, there is no need to implement various versions of programming logic
to reason about different kinds of properties with various code annotations.
On the other hand, we provide automation libraries that assist reasoning on various common properties at the logical level. 


\section{OS Kernel Verification}

Bevier~\cite{bevier89} developed a full correctness proof
for a highly idealized kernel in an automated theorem prover. The
Verisoft team~\cite{verisoft07} has done a large body of work aiming
to verify OS kernels and
hypervisors~\cite{leinenbach09,alkassar10}. The Verve
project~\cite{hawblitzel10} managed to prove the type safety of an
entire kernel by combining the partial correctness proof of a nucleus
and the type-safety guarantee from a certifying C\# compiler (for the
rest of the kernel); by using powerful automated proving tools (e.g.,
Boogie and Z3), Verve managed to certify the nucleus in 9
person-months.
Xu~{\em et al} \cite{xu16} developed a new verification framework based on RGSim
and Feng~{et~al.}'s program logic~\cite{feng08:aim} for reasoning
about interrupts; they have successfully verified many key modules
(in C) in the $\mu$C/OS-II kernel.

The seL4 team~\cite{klein2009sel4} was the first to verify the
correctness and security properties of a high-performance L4-family
microkernel. Their work is impressive in that most of their proofs
were done inside a modern mechanized proof
assistant~\cite{Paulson:Isabelle}.  To make verification easier, they
introduced an intermediate executable specification to hide C
specifics. They claim that kernel
interdependencies led to more complex invariants which consist
of majority of their proofs. seL4 also does not support multicore
concurrency with fine-grained locking. 
We believe that separating out the complex invariants and concurrency
reasoning from the actual code verification could significantly reduce
the proof complexity and reduce the amount of repeated proofs.

Hawblitzel~{\em et al}~\cite{ironclad14} has developed a set
of new tools based on the Dafny verifier~\cite{dafny10} and Z3 SMT
solver~\cite{moura08}, and applied them to build their Ironclad system
which includes a verified kernel (based on Verve~\cite{hawblitzel10}),
verified drivers, verified system and crypto libraries, and several
applications.  This is another impressive effort that advances the
frontier of system software verification. Ironclad, however, only
proves the partial correctness property (at the assembly level), which
is weaker than the total correctness properties proved by seL4 and
mCertiKOS. 
Their device models are also at a very high level, thus many low level
drivers are not verified. Ironclad also differs
from seL4 and mCertiKOS in that its proofs are all done by an SMT solver
which does not produce any machine-checkable proof objects.

\section{Verification of Device Drivers and Interrupts}

Gu {\em et al} \cite{dscal15} pioneered the compositional
proof machinery that
builds certified OS kernels using deep specifications and certified
abstraction layers. We built our certified interruptible OS kernel and
device drivers using the same methodology. Our new compositional proof
framework, however, adds two novelties.
%%%%%%%%%%%%%%%%
First, we show how to handle
device objects, which are different from regular mCertiKOS kernel
objects. The states in these new device objects can be updated either
by the kernel (via device methods) or by an external environment,
whereas regular mCertiKOS objects can only be mutated synchronously by
the CPU; device objects can also be asynchronously mutated by the
environment; we introduce a new abstraction of per-device event logs
to handle this asynchronicity.
%%%%%%%%%%%%%%%%
Second, we support formal reasoning about kernel code and device
drivers running on multiple logical CPUs (see
Fig.~\ref{fig:layer_new}) while under Gu {\em et al} \cite{dscal15}, all verified
code at each layer must run on a single CPU (see
Fig.~\ref{fig:layer_pre}); we treat the driver stack for each device
as if it were running on the logical CPU dedicated to that device.

Klein {\em et al} \cite{klein2009sel4} were the first to verify the correctness and
security properties of a high-performance L4-family microkernel in a
modern mechanized proof assistant~\cite{Paulson:Isabelle}.  To make
verification easier, they introduced an intermediate executable
specification to hide C specifics. Gu {\em et al} \cite{dscal15} built their
certified mCertiKOS kernel (in Coq) by decomposing it into many
abstraction layers; such fine-grained layer decomposition led to
significantly lower proof and development effort and also better
extensibility. Both kernels, however, lack a realistic
interrupt model, so reasoning about interruptible code is not
supported. The device drivers are not verified in either kernel.

Hawblitzel~{\em et al}~\cite{ironclad14} has recently developed a set
of new tools based on the Dafny verifier~\cite{dafny10} and Z3 SMT
solver~\cite{moura08}, and applied them to build their Ironclad system
which includes a verified kernel (based on Verve~\cite{hawblitzel10}),
verified drivers, verified system and crypto libraries, and several
applications.  This is another impressive effort that advances the
frontier of system software verification. However, the abstract device
model in Ironclad is too high level to model many hardware details.

The Verisoft team~\cite{verisoft07} has done a large body of work
aiming to verify an OS kernel with device drivers in a proof assistant
\cite{Alkassar:OSVE09,Alkassar:VSTTE08-225,Alkassar:VSTTE2010-71}.
Alkassar and Hillebrand \cite{Alkassar:VSTTE08-225} reported
their work on verification of device
driver, which investigated the relationship between external events, device
and processor execution, and proved several non-trivial lemmas that can be
viewed as a foundation of this work. In their work, the execution of CPU and
devices are modeled as a combined transition system with an oracle to select
the next one to make a step. The transition of each step is categorized into
three cases: (1) processor-device transition, (2) local processor transition,
and (3) external device transition. The paper shows that: a. the steps of
local processor and external device transition can be swapped because they do
not interfere with each other; b. the steps of external device transition by
other devices can be reordered to the end of the execution because they do not
change the state of the current device with the assumption that all the
drivers only access to their own devices. Therefore, the device transitions
triggered by external events can be moved from the code of high-level language
which do not access the devices into the driver code which is written in
assembly. In addition, the specification of the driver code which involves
processor and device steps can be called atomically. They further proved the
correctness of a simple ATAPI disk driver.
In our paper, the ``logical'' CPUs are systematically built in an isolated way,
so that the assumption that drivers only access to their own devices are
guaranteed by construction. Furthermore, viewing drivers as more abstract devices
enables us to add device accessing primitives into the device “ISA”, and
encapsulate the transitions of devices layer by layer into richer yet atomic
specifications. Last, interrupt is allowed when the driver is running. We use
\textsf{intr\_disable} / \textsf{intr\_enable} to mark the specific regions
as critical sections,
so that the interrupt being disabled is not an assumption of our driver code.

Based on the ideas of concurrent separation logic (CSL), Alkassar {\em el at}
\cite{Alkassar:IPC}
presented a modular and polymorphic method to specify IPC algorithms in VCC,
which relies on transferring ownership of the ghost objects between IPC
entities and attached invariants to guarantee the correctness of the
algorithm. They extended their specification pattern to specify and verify an
inter-processor interrupt (IPI) protocol in multiprocessor systems, in which
IPC mailboxes are used to model the APIC bus between processors, and the IPI
sending and receiving (NMI interrupt handler) code are modeled as while-
loops. They focused on the interaction between the IPI participants locally,
but not between the triggered interrupt handler and the previous execution on
its CPU. In our device driver verification, we also address the interleavings
between normal execution and interrupt handler on the same processor.
In addition, the verification in \cite{Alkassar:IPC} does not
prove strong contextual refinement property as in our paper.

The idea of shuffling the execution of interrupt handlers to a certain point
so that the verification can be done at higher level languages is discussed in
Pentchev's PhD thesis \cite{Pentchev:2016}. Their concurrent machine $MIPS_{P}$
was modeled as an
automaton with a sequence of step-indicators as external inputs. At each step,
the transition function makes a case distinction based on an external input
pair (component i.e. core / ipi / guest / vmexit, processor) and let one or
multiple components to make a step. In order to propagate properties from the C
Intermediate Language (C-IL) program to the $MIPS_{P}$ execution, they applied
order reduction to prove the sequential compiler consistency (a simulation
relation to achieve that the execution of interrupt handlers only happens
between the C statements) by having interleaving occur at consistent state,
namely interleaving points. In contrast to our paper, because the IPIs are
non-maskable, the interrupt handler has to be carefully designed to avoid the
race condition, and they only provided the semantics of concurrent C-IL
encapsulated with IPI to specify the interrupt handler. In our paper, we push
it further to the boundary of critical sections. We do not consider the
interleaving between the kernel and device execution at such early stage. We
view them as separated logical machine and design their own transition functions.
In addition, the states among logical CPUs are isolated and are not visible
to each other. The consistency of the states between a driver and its
interrupt handler is guaranteed by examining the same local log to construct
all shared states. We enforce a principle of a common programming pattern that
drivers have to disable the interrupt before entering into the critical section,
so that the ``interleaving points'' in our paper is only the \textsf{intr\_disable} and
\textsf{intr\_enable}. Thus, above a certain layer, the code verification can not only
be done at Clight level, but also be free of considering the interleaving between
the code and interrupt handlers, because they are all encapsulated in the primitive of
those functions accessing shared states.

Feng {\em et al} \cite{feng08:aim,feng09:jar} developed a formal Hoare-logic-like
framework for certifying low-level system programs involving both
hardware interrupts and preemptive threads.  Using ideas from
concurrent separation logic~\cite{ohearn:concur04}, they showed how to
use ownership-transfer semantics to model enabling and disabling
interrupts and reason about the interaction among interrupt handlers,
context switching, and synchronization libraries.  They successfully
certified a preemptive thread implementation (as libraries) and a set
of common synchronization primitives in the Coq proof assistant.
Their work, however, did not model any hardware device or interrupt
controller, and their interrupt model is much simpler than ours. They
also only proved the partial correctness property (for their certified
library functions), not the strong contextual refinement property
which we proved for our kernel. Of course, since our current certified
kernel does not support preemptive concurrency, we believe there are
good opportunities for combining their techniques (for reasoning about
preemptive concurrency) with our refinement-based approach.

Ryzhyk {\em et al} \cite{Ryzhyk_09,Ryzhyk14} have done much work on
the synthesis of device drivers from the specifications.
In their approach, both the device and the interface of the corresponding
driver are modeled as state machines, which communicate via messages. 
The generated driver code requires some unverified run-time
support. Furthermore, the correctness of the drivers is limited to the
synthesized C programs, not the compiled assembly code running on the actual
hardware. 

In the work of Duan and Regehr \cite{Duan2010}, a UART driver in the
ARM architecture with interrupt is verified. They have created an
abstract device model which gets plugged into the instruction set of
the ARM6 architecture.  In their model, the device state is mixed into
the machine state. Thus, they have to carefully consider the
interleavings between the execution of the device and the CPU. Albeit
a realistic UART model, the driver only consists of 20 lines of the
assembly code. The framework is later ported to the Cambridge model of
the ARMv7 architecture \cite{duan2013}.  Schwarz {\em et al}
\cite{Oliver2014} proposed a device model where all the devices are
executed nondeterministically in parallel with a single core
processor. Based on the model, they have proved several
noninterference properties among the processor and devices which
potentially use DMA or interrupts.  Monniaux {\em et al}
\cite{Monniaux_EMSOFT07} have verified a driver with a USB OHCI
controller model written in C with a static analyzer.  They have
showed the verified driver exhibits no undefined behavior.

Andronick {\em et al} \cite{Andronick2015, Andronick2016} presented
a scalable framework for formally reasoning an
embedded, real-time operation system: eChronos. In eChronos, OS functions are
treated as interrupts, which are wrapped with supervisor calls (software triggered
interrupts) even in the same address space. An extended Owicki-Gries approach
is used to model and verify the system as all the concurrency components,
such as tasks, interrupt handlers, supervisor calls, and asynchronous hardware
behaviors, which are composed in parallel. Furthermore, in order to show the
controlled interleaving allowed by the hardware, a ghost variable ``AT'' is
introduced to represent the current active task. The correctness of eChronos
is proven by properties (expressed by invariants) held through all
executions and at every reachable step.  Because of the non-determinism
introduced by parallel composition, it is challenging to write the
specification in an atomic way so that refinement can be applied to prove the
functional correctness.

There are many lines of work in verifying device drivers based on
model checking. Amani {\em et al} \cite{Amani12} proposed an approach
to automatically verify the protocols between drivers and the
operating system.  Thomas Witkowski \cite{witkowski2007} and Alexey
Khoroshilov \cite{Khoroshilov2010} have verified specific protocols of
some Linux drivers using the model checker \textsc{SatAbs} and
\textsc{DDVerify}.  Kim {\em et al} \cite{Kim2008} have verified a
driver for a flash memory in NuSMV, Spin, and CBMC.  Ball {\em et al}
\cite{slam2} have developed the static analysis tool SLAM, which is
included in the Microsoft Windows Driver Developer Kit.
