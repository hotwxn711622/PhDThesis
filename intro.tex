An operating system (OS) kernel serves as the lowest level of any
system software stack. The correctness of the OS kernel is the basis
for that of the entire system. In a monolithic kernel, device drivers
form the majority of the code base; 70\% of the Linux 2.4.1 kernel are
device drivers~\cite{Chou:2001}. Furthermore, such drivers are found
to be the major source of crashes in the Linux and Windows operating
systems~\cite{Chou:2001,Ball:2006,Ganapathi:2006}. While recent
efforts on seL4~\cite{klein14} and CertiKOS~\cite{dscal15} have
demonstrated the feasibility of building formally verified OS kernels,
it is unclear how to extend their work to verify the functional
correctness of device drivers. In CertiKOS~\cite{dscal15}, drivers are
unverified, and it is not obvious how to extend their framework to
model devices and interrupts. In a microkernel like
seL4~\cite{klein14}, device drivers are implemented in user space,
and, though its proofs guarantee driver isolation, it does not
eliminate bugs in its user-level drivers.
% The Verisoft team~\cite{verisoft07} has done a large body of work
% aiming to verify an OS kernel with device drivers
% \cite{Alkassar:OSVE09,Alkassar:VSTTE08-225,Alkassar:VSTTE2010-71},
% but their drivers are verified completely at the assembly level.
% Furthermore, they put many restrictions on the devices they can
% verify, many of which are not satisfied by a large number of
% existing devices.

A major challenge in driver verification is the interrupt: a non-local
jump to some driver code, triggered by a device. When device drivers
are implemented inside the kernel (for better performance), the kernel
should be interruptible; otherwise, it can lead to an unacceptable
interrupt processing latency.  Reasoning about interruptible code is
particularly challenging, since every fine-grained processor step
could contain a non-local jump, and, upon return, the machine state
could be substantially changed. Even worse, it is not clear how such
reasoning should be done at the C level, which is completely
interrupt-unaware. Existing work either assumes that interrupts are
turned off inside the kernel~\cite{dscal15,verisoft07}, or polls the
interrupts at a few carefully chosen interrupt points~\cite{klein14}.

Furthermore, interrupt hardware is not static, but is configured by
software. In order to verify any interesting device drivers (serial,
disk, {\it etc}.), we first need to model the interrupt controller
devices (e.g., LAPIC~\cite{mps97}, I/O APIC~\cite{ioapicd96}), and
formally verify their drivers. This is important because, if the
interrupt controllers are not initialized properly, it may lead to
undesired interrupt behaviors. Device drivers also interact with
interrupt controllers to mask/unmask particular interrupt lines.
These issues have been overlooked in past work, where 
interrupt controllers are assumed to be properly initialized and
their drivers are correctly implemented~\cite{Alkassar:VSTTE08-225}.

Finally, verifying an interruptible operating system with device
drivers also faces the following challenges.

{\em Devices and CPU run in parallel.} Thus, the executions of CPU
instructions and device transitions can interleave arbitrarily. Code
verification on this highly nondeterministic machine can be
challenging, since it needs to consider device state transitions, even
when the CPU is executing a set of instructions unrelated to external
devices. Recent
work~\cite{Alkassar:OSVE09,Alkassar:VSTTE08-225,Alkassar:VSTTE2010-71}
tries to address this by enforcing a {\it stability} requirement
that device states only change due to CPU operations. 
%
% Using this assumption, they have proven a reordering
% theory and applied it to postpone the execution of all CPU instructions
% unrelated to the device that is currently being manipulated.  They are then
% able to turn the set of assembly level device operations into
% atomic primitives, which are then lifted to their C level verification
% framework as inline-assembly calls~\cite{Alkassar:VSTTE2010-71}.
% Though the stability assumption suffices for their use case, it is
%
This requirement is, however, 
too strong as devices interacting with 
external environments are not stable: a serial device constantly
receives characters through its port, a network card continuously
transfers packets, an interrupt controller (IC) asynchronously
receives interrupt requests, {\it etc}.

{\em Devices may directly interact with each other.} Existing work
assumes that a device driver monopolizes its underlying device and
devices do not influence each other~\cite{Alkassar:VSTTE08-225}. This
assumption does not hold for many devices in practice. For example,
most devices directly communicate with an interrupt controller by
signaling an interrupt.

{\em Device drivers are written in both assembly and C.}  Existing
device driver verification is either done completely at the assembly
level \cite{Alkassar:VSTTE08-225,duan2013} or the verified properties
are only guaranteed to hold at the C level \cite{Ryzhyk_09,Ryzhyk14}.
For realistic use-cases, proven properties should be 
translated down and then formally linked with
the assembly-level proofs.

% {\em The correctness of device drivers needs to be formally connected
%   to the correctness of the OS kernel in a single proof
%   context.}
%
{\em The correctness results of different components should be
integrated formally.} For example, the correctness proofs of device
drivers and the OS kernel
% , both written in a combination of C and assembly,
need to be formally linked as an integrated system,
before one can deliver formal guarantees on the OS as a whole. Not
doing so can introduce semantic gaps among different modules,
% wherein a module makes assumptions contradicting with those of another,
a scenario which introduced actual bugs in previous verification
efforts as reported by Yang and Hawblitzel~\cite{hawblitzel10}. Unfortunately,
this formal linking process was found to be even more challenging than
the correctness proofs of individual modules
themselves~\cite{Alkassar:VSTTE08-225}. Even OS's with user-level
device drivers can suffer if the correctness proofs of their drivers are not
formally linked with those of the kernel. For example,
if some device driver code triggers a page fault at the user level,
the behavior of the corresponding driver is linked to the behaviors of
the page-fault handlers and address translation mechanism of the kernel.

In this paper, we propose a novel compositional approach that tackles
all of the above challenges. There are two key contributing ideas.
One is to build up a certified ``virtual'' device hierarchy, and
the other is a new abstract interrupt model, built upon a realistic
hardware interrupt model through contextual refinement. We use these
to build an extensible framework that systematically enforces the
isolation among different operating system modules, which is important
for scalability of any verification effort and critical for reasoning
about interruptible code.

%%%%%%%%%%%%%%%%%%%%%%%%%%%%%%%%%%%%%%%%%%%%%%%%%%%%%%%%%%%%%%%%%%%%%%%%%%%%
% TODO: (1) revise the following list of contributions
%       (2) a new statement about our contributions over Gu et al
%       (3) outline of the rest of the paper
%       (4) refer to a later section regarding limitations
%%%%%%%%%%%%%%%%%%%%%%%%%%%%%%%%%%%%%%%%%%%%%%%%%%%%%%%%%%%%%%%%%%%%%%%%%%%%

Our paper makes the following new contributions:
\begin{itemize}
\item We present a new extensible architecture for building certified
  OS kernels with device drivers. Instead of mixing the device drivers
  with the rest of the kernel (since they both run on the same
  physical CPU), we treat the device drivers for each device as if
  they were running on a ``logical'' CPU dedicated to that device.
  This novel idea allows us to build up a certified hierarchy of
  extended abstract devices over the raw hardware devices, meanwhile,
  systematically enforcing the isolation among different ``devices''
  and the rest of the kernel.

%%%%%
\item We present a novel abstraction-layer-based approach for
  expressing interrupts, which enables us to build certified
  {\em interruptible} OS kernels and device drivers. Our formalization of
  interrupts includes a realistic hardware interrupt model, and an
  abstract model of interrupts which is suitable for reasoning about
  interruptible code. We prove that the two interrupt models are
  contextually equivalent.

%%%%%
\item We present, to the best of our knowledge, the first verified
  interruptible OS kernel and device drivers that come with
  machine-checkable proofs.  The implementation, modeling, specification, and
  proofs are all done in a unified framework (realized in the Coq
  proof assistant~\cite{coq}), yet the machine-checkable proofs verify the
  correctness of the assembly code that can run on the actual
  hardware.
\end{itemize}

The rest of this thesis is organized as follows.
Chapter~\ref{chapter:framework} presents our abstraction-layer-based verification
technology using a concrete example, while
Chapter~\ref{chapter:automation} details the challenges and solutions to
our automation engine for verifying C programs.
In Chapter \ref{chapter:sequential}, we present the concrete Coq
implementations of the mCertiKOS verified
kernel using the techniques presented in the previous chapters.
The framework is further extended in Chapter \ref{chapter:driver}
to support modular reasoning of device drivers and interrupts.
A modular framework that can further support shared-memory concurrency
with fine-grained locking is presented in Chapter \ref{chapter:concurrent}.
Then, the limitations of various frameworks are discussed in
Chapter \ref{chapter:limitation}, and we discuss various important
related works in Chapter \ref{chapter:related}, and finally conclude
in Chapter \ref{chapter:conclusion}.

\paragraph{My Contributions}
The work described in this thesis is based on joint work with various members
in the CertiKOS project.\footnote{http://flint.cs.yale.edu/certikos/}
The author is one of the two critical contributors for developing the methodology
on performing large scale verification by building stacks of certified abstraction
using two common patterns, presented in Chapter \ref{chapter:framework}. The automation
engines for proving the C source programs (described in Chapter \ref{chapter:automation})
are developed solely by the author, which is particularly suitable for the
layer-based methodology and is the critical component of the framework. 
The author is also one of the two main contributors of the verified kernel (presented
in Chapter \ref{chapter:sequential}), and the
extension to the framework to reason about device drivers and interrupts
(detailed in Chapter \ref{chapter:driver}), and a critical contributor
of the log-based approach on reasoning
concurrent programs (illustrated in Chapter \ref{chapter:concurrent}). 

\ignore{
Chapter \ref{chapter:concurrent}
briefly explains the ideas of certified concurrent abstraction layers
and presents some of the key components the author contributed
for the verified concurrent kernel: verification of ticket lock algorithm,
and extending the verified hypervisor to support multiple virtual machines
(one virtual machine on each CPU core). 
the hardware device models and verification of the device
driver C source code using the automation engines the author developed,
proving the refinement of
layer interfaces and interrupt models and the driver implemented in assembly. 
}
