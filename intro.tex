While safety-critical software is often designed and tested with extra efforts, testing cannot
prove the absence of bugs. Formal verification, despite known as the only way of building
bug-free software systems, has not been exercised widely in practice due to its complexity and cost.
Verification of a reasonably complex system software is known to be extremely more complex and
costly than the development of the software itself~\cite{klein2009sel4,klein14}.


\section{Challenges in Building Formally Verified System Software}

This dissertation aims to provide a very effective and accessible way of proving practical low-level
system software. To achieve this, we need to tackle several main challenges.

First, complex system software consists of highly interdependent subsystems.
The modular reasoning of each small component locally is non-trivial.
Separation logic is one of the most commonly
used tools to reason about the isolation of data structures in the memory
\cite{appel07:tphols,Tuch:2009}. 
\ignore{
On the other hand, it mixes the main functional
correctness logic with the proof
of memory isolation, thus lacks in modularity and hinders opportunities for automation.
}
As a success story,
the seL4 team \cite{klein2009sel4} has developed the first
formally verified practical microkernel.
This work is impressive in that all the proofs were done
in a modern mechanized proof assistant.
On the other hand, it took them 11 person-years to verify 7,500 lines of
C code, still leaving 1,300 lines of C and 500 lines of assembly code
unverified.


Next, the majority of the traditional verification efforts on proving system software go into proving invariant preservation properties.
For example, about 80\% of the properties proved in seL4 are related
to preserving invariants \cite{klein14}. 
Invariants are known to be very expensive
because we not only need to prove that they are preserved by the local function in consideration,
but also for the whole system, i.e., we need to show that the invariants are preserved at any
moment of the whole system execution and cannot be accidentally broken by any functions
in the system, e.g., through pointer manipulation. Even worse, in practice, we normally do
have functions that temporarily break the invariants. For example, an operation to a data structure
that well preserves the invariants could be implemented with multiple auxiliary functions, where
some of these functions temporarily violate the invariants while some re-establish
the invariants later. Traditional verification approaches
do not have a very effective and systematic way of tackling this challenge.

Furthermore, 
in practical system software, device drivers often form a very large component
of the entire code base; 70\% of the Linux 2.4.1 kernel is
device drivers~\cite{Chou:2001}.
Such drivers are found
to be the major source of crashes~\cite{Chou:2001,Ball:2006,Ganapathi:2006}.
A major challenge in driver verification is the interrupt: a non-local
jump to some driver code, triggered by a device. 
Reasoning about interruptible code is
particularly challenging since every fine-grained processor step
could contain a non-local jump, and, upon return, the machine state
could be substantially changed. Even worse, it is not clear how such
reasoning should be done at the C level, which is completely
interrupt-unaware. Existing work either assumes that interrupts are
turned off inside the kernel~\cite{dscal15,verisoft07} or polls the
interrupts at a few carefully chosen interrupt points~\cite{klein14}.

Last not the least, many system software run on multiple CPUs with
multiple concurrent threads.   
The efficient reasoning of arbitrary concurrent executions among various
interdependent components in a complex low-level system is considered
intractable by some researchers~\cite{vontessin13,peters15,Chong:2016:RNW:3040225}. 

\section{A Compositional Automation Engine} 

In this thesis, we present a compositional, and powerful automation engine for effectively
verifying complex system software. 
We believe that the aforementioned challenges are not directly tied to the verification
of the functional correctness itself, and we think the automation engine should focus solely
on providing
very strong automation support for proving the functional correctness of each system component,
completely separate from the reasoning of aforementioned challenges.
We have developed a strong automation tactical libraries that solely focus on
proving functional correctness, and provide a separate logical framework to handle
memory isolation, invariant preservation, device interrupts, and concurrency.

\paragraph{Memory Isolation}
Instead of letting the main functional correctness proof handle the low-level
isolation at the memory level,  we provide a systematic approach to abstract in-memory
data structures into equivalent logical abstract states where the isolation is guaranteed.
For every in-memory data structure, before we verify the corresponding functions operating
on the data, we first develop a new equivalent machine where the in-memory data is gone but replaced
with a logical abstract state. This new machine provides very basic read/write primitives to this
abstract state, which are implemented by simple getter/setter functions in the original machine.
These two machines, what we call {\it abstraction layers}, are proved to be {\it contextually equivalent},
where any program running on the more abstract machine would have the same behavior when it gets
linked with the memory getter/setter code and  run on the other machine.
At the newer abstraction layer, the read/write primitives are the only way to access the new abstract
states, thus the abstract states are completely isolated from the rest of in-memory data and other
abstract states, and cannot be accessed by other primitives or memory operations.
The actual functions, which are implemented on top of the new abstraction layer
using the new read/write primitives, can be verified
effectively with our automation engine completely at the logical level, and we no longer need to deal with
low-level memory or isolation. Here, a new abstraction layer is developed where the verified
functions are turned into logical primitives with their specifications, and new functions can be
further verified on top of this new abstraction layer.

\paragraph{Invariant Preservation}
As we abstract in-memory data structures into logical abstract states, our system invariants
are also expressed in terms of the abstract states. At every abstraction layer, we separately
prove that all the primitives in this layer satisfy all the invariants in this layer. Given that these
primitives are the only way to access the abstract states, by design, we know that any piece of
code running on this abstract machine model would not break the invariants. 
Different abstract states are isolated, thus by design, primitives operating on one abstract
state would not violate invariants imposed on other abstract states.
Furthermore, each
layer could have different sets of invariants, thus we no longer need to introduce entire system
invariants at once, but introduce them gradually at the most appropriate abstraction level.
For example, when we only have auxiliary list operations that break the list integrity invariants,
we can postpone introducing these invariants until we get to the layer where we have
verified the full functions that gracefully operate on the list preserving the invariants.
The intermediate auxiliary primitives that are no longer needed are removed from the new abstraction
layer to make sure the new invariants are preserved, or to avoid unnecessary invariants proof
even when they do preserve the new invariants added.

\ignore{
To experiment with the effectiveness of our automation engine, 
we have successfully verified a practical single-core
operating system kernel with rich features like virtual
memory, process management, hardware virtualization, {\it etc}.
The proof is achieved semi-automatically with our automation
engine implemented in Coq proof assistant and is detailed
in Chapter \ref{chapter:sequential}.
}

\paragraph{Proof Automation}
As the automation engine guarantees the memory isolation and the invariant
preservation, we can take these as given while we perform the functional
correctness proof. Further more, the main complex functional correctness
reasoning can now be performed on logical abstract states with logical
primitives. Based on this, we have developed a collection of strong
automation tactic libraries that can be used to automate the
proof for varieties of C source programs.

\paragraph{Devices and Interrupts}

In addition to the main challenge brought by the device interrupt,
verification of an interruptible system software with device
drivers also faces the following challenges.

{\em interrupt hardware is not static.} It is configured by
software. In order to verify any interesting device drivers (serial,
disk, {\it etc}.), we first need to model the interrupt controller
devices (e.g., LAPIC~\cite{mps97}, I/O APIC~\cite{ioapicd96}), and
formally verify their drivers. This is important because, if the
interrupt controllers are not initialized properly, it may lead to
undesired interrupt behaviors. Device drivers also interact with
interrupt controllers to mask/unmask particular interrupt lines.
These issues have been overlooked in past work, where 
interrupt controllers are assumed to be properly initialized and
their drivers are correctly implemented~\cite{Alkassar:VSTTE08-225}.

{\em Devices and CPU run in parallel.} Thus, the executions of CPU
instructions and device transitions can interleave arbitrarily. Code
verification on this highly nondeterministic machine can be
challenging since it needs to consider device state transitions, even
when the CPU is executing a set of instructions unrelated to external
devices. Recent
work~\cite{Alkassar:OSVE09,Alkassar:VSTTE08-225,Alkassar:VSTTE2010-71}
tries to address this by enforcing a {\it stability} requirement
that device states only change due to CPU operations. 
%
% Using this assumption, they have proven a reordering
% theory and applied it to postpone the execution of all CPU instructions
% unrelated to the device that is currently being manipulated.  They are then
% able to turn the set of assembly level device operations into
% atomic primitives, which are then lifted to their C level verification
% framework as inline-assembly calls~\cite{Alkassar:VSTTE2010-71}.
% Though the stability assumption suffices for their use case, it is
%
This requirement is, however, 
too strong as devices interacting with 
external environments are not stable: a serial device constantly
receives characters through its port, a network card continuously
transfers packets, an interrupt controller (IC) asynchronously
receives interrupt requests, {\it etc}.

{\em Devices may directly interact with each other.} Existing work
assumes that a device driver monopolizes its underlying device and
devices do not influence each other~\cite{Alkassar:VSTTE08-225}. This
assumption does not hold for many devices in practice. For example,
most devices directly communicate with an interrupt controller by
signaling an interrupt.

{\em Device drivers are written in both assembly and C.}  Existing
device driver verification is either done completely at the assembly
level \cite{Alkassar:VSTTE08-225,duan2013} or the verified properties
are only guaranteed to hold at the C level \cite{Ryzhyk_09,Ryzhyk14}.
For realistic use-cases, proven properties should be 
translated down and then formally linked with
the assembly-level proofs.

% {\em The correctness of device drivers needs to be formally connected
%   to the correctness of the OS kernel in a single proof
%   context.}
%
{\em The correctness results of different components should be
integrated formally.} For example, the correctness proofs of device
drivers and the rest of the system
% , both written in a combination of C and assembly,
need to be formally linked as an integrated system,
before one can deliver formal guarantees on the system as a whole. Not
doing so can introduce semantic gaps among different modules,
% wherein a module makes assumptions contradicting with those of another,
a scenario which introduced actual bugs in previous verification
efforts as reported by Yang and Hawblitzel~\cite{hawblitzel10}. Unfortunately,
this formal linking process was found to be even more challenging than
the correctness proofs of individual modules
themselves~\cite{Alkassar:VSTTE08-225}. 
\ignore{Even OS's with user-level
device drivers can suffer if the correctness proofs of their drivers are not
formally linked with those of the kernel. For example,
if some device driver code triggers a page fault at the user level,
the behavior of the corresponding driver is linked to the behaviors of
the page-fault handlers and address translation mechanism of the kernel.}

Following our mantra, we isolate all the above challenges from our main
functional correctness proof such that we can reuse all the techniques
we have developed above for effective verification of device drivers.
To reason about isolation among concurrent devices and the other parts
of the system, we build up a certified ``virtual'' device hierarchy where
the isolation among different ``devices'' are guaranteed by design.
For effective reasoning of interruptible code, we build a new abstract
interrupt model, built upon a realistic
hardware interrupt model through contextual refinement.
As a result, we have built an effective system that systematically enforces the
isolation among different system modules, which is important
for the scalability of any verification effort and critical for reasoning
about interruptible code. 
\ignore{
We have successfully used these techniques to build the world's first fully verified
interruptible OS kernel with device drivers.
The details are presented in Chapter \ref{chapter:driver}.
}

\paragraph{Reasoning about Concurrent Programs}

In concurrent programs, the accesses to the shared states are protected, either by
locks or via lock-free algorithms. On the other hand, in the developers' mind, these accesses
are considered ``atomic'', and this atomicity needs to be reflected in any good specifications.
When we reason about these concurrent programs, the challenges come from the fact
that we need to consider all possible interleaved executions between the current program
and any unknown potential environment programs and prove that even with all these interleaved
executions, the concurrent accesses can be safely shuffled such that we can view as if
the current program accesses the shared states atomically, and this behavior is observably
equivalent to the original interleaved executions.

Again, following our mantra, we have developed an effective way of representing the states
shared among multiple concurrent programs and expressing the invariants on the current program
and assumptions for other companion concurrent programs. While we continue to abstract private in-memory
states into logical abstract states, the shared states are represented instead as a list of globally
observable I/O events, where actual states can be inferred given the list of events. The operations
on the shared states by the current program are specified as appending corresponding events to the
global external log list. The operations on the shared states by other concurrent programs are encapsulated
into the {\it environmental context}, which specifies the shared state changes by providing a list of events
generated by other concurrent programs, under the current concurrent execution context.
By achieving the proof under all possible environmental context, we could conclude that
the verified property holds for any possible interleaved executions among the concurrent programs.
On the actual verification of functional correctness, we always assume all the accesses to the shared
states are protected, e.g., the lock is already held, such that we no longer need to reason about interleavings
during the code verification. The reasoning of interleaved executions is done as a separate next step,
again at a completely logical level.
This allows us to avoid any complexities coming from the concurrency into our automation engine,
and effectively reuse all the existing
verification techniques to verify each concurrent program, and effectively
combine the reasoning into global properties on the combined concurrent programs.
\ignore{
This line of work is presented in Chapter \ref{chapter:concurrent}.
}

\section{Key Contributions}

\begin{itemize}
\item To show the effectiveness of our automation engine, we have
successfully developed a fully verified
practical feature-rich certified OS kernel, which runs
on stock x86 hardware and doubles as a hypervisor and boots a version of Linux as the guest.
We managed to achieve the full specification and verification of the kernel within
2 person-years.

\item With the support of devices and interrupts, 
we have developed, to the best of our knowledge, the world's first verified
interruptible OS kernel with device drivers that come with
  machine-checkable proofs. This work is built on top of the verified kernel above
  and entire verification effort took less than one additional year.
  
\item With the extension to support concurrency, we have
successfully extended the kernel further
into the world's first fully verified concurrent OS kernel with fine-grained locking.
The entire verification effort took two additional person-years.

\item In all verified systems, the implementation, modeling, specification, and
  proofs are all done in a unified framework (realized in the Coq
  proof assistant~\cite{coq}), yet the machine-checkable proofs verify the
  correctness of the assembly code that can run on the actual
  hardware.
  
\end{itemize}

\section{Contributions by Collaborators}

The work described in this thesis is based on joint work with various members
in the CertiKOS project.\footnote{http://flint.cs.yale.edu/certikos/}
The author collaborated with Ronghui Gu on developing the sequential certified abstraction layers
(presented in Chapter \ref{chapter:framework}) and the fully verified
sequential OS kernel (detailed in Chapter \ref{chapter:sequential}), where Ronghui's focus was on the refinement proof
while the author focused on the code verification.
The automation
engine for proving the C source programs (described in Chapter \ref{chapter:automation})
are developed solely by the author, which is particularly suitable for the
layer-based methodology and is the critical component of the framework.
The work on the framework to reason about device drivers and interrupts
(detailed in Chapter \ref{chapter:driver}) are also contributed
by Hao Chen, who managed to develop all the device models and performed the code verification
while the author focused on the refinement proof, in particular, developing the abstract interrupt model
that is suitable to reason about interruptible code.
The author also worked together with Ronghui Gu and Jieung Kim on the development of
concurrent certified abstraction layers and the verified concurrent OS kernel
(illustrated in Chapter \ref{chapter:concurrent}).


\section{Summary of My Contributions}

\begin{itemize}
\item We present a novel, compositional, and powerful automation engine
for verifying C source code against their specifications. Unlike traditional
verification approaches for verifying C programs mixing the proof of memory isolation
and preservation of invariants into the reasoning of functional correctness,
our automation engine takes the memory isolation and invariants as given, and provide
separate mechanisms to guarantee isolation and prove the invariants.
This allows us to provide very strong automation support focusing solely on
proving the functional correctness.


\item We have extended the engine with a novel way of representing
devices and drivers, to effectively reason about system software
with device drivers.
Instead of mixing the device drivers
  with the rest of the system (since they both run on the same
  physical CPU), we treat the device drivers for each device as if
  they were running on a ``logical'' CPU dedicated to that device.
  This novel idea allows us to build up a certified hierarchy of
  extended abstract devices over the raw hardware devices, meanwhile,
  systematically enforcing the isolation among different ``devices''
  and the rest of the kernel.

\item We present a novel abstraction-layer-based approach for
  expressing interrupts, which enables us to build certified
  {\em interruptible} system software with device drivers. Our formalization of
  interrupts includes a realistic hardware interrupt model and an
  abstract model of interrupts which is suitable for reasoning about
  interruptible code. We prove that the two interrupt models are
  contextually equivalent.
  
\item We have further extended our engine with an effective way of representing
shared states accessed by concurrent programs and making assumptions
on other concurrent programs. This allows us to reuse all of the
existing verification techniques to verify concurrent programs, assuming as if
they run sequentially, and move all the reasoning of interleaved executions
to a separate logical reasoning framework.  

  
\end{itemize}

\section{Acknowledgment}

This research is based on work
supported in part by DARPA grants FA8750-10-2-0254, FA8750-
12-2-0293, FA8750-16-2-0274, and FA8750-15-C-0082 and NSF grants 1065451, 0915888, 1521523, and 1319671 and ONR Grant
N00014-12-1-0478. 
Any opinions, findings,
and conclusions contained in this document are those of the authors
and do not reflect the views of these agencies.

