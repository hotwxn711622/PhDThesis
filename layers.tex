\section{Coq Implementation of Certified Abstraction Layers}
\label{sec:layers}

In this section, we present in detail the actual implementation of the
certified abstraction layers. To make it more easy to reproduce our work,
various examples in this sections are presented directly as they are
implemented in the Coq proof assistant. Various Coq constructions commonly
used throughout this section is presented in Table \ref{table:coq}.
We also explain them as they show up in the examples.

Starting from a bottom-most layer interface abstracting the CPU, device
models and external interfaces, we gradually introduce driver code modules
to develop certified abstraction layers by introducing higher level layer
interfaces abstracting the concrete driver behaviors, and showing contextual
refinement between each overlay interface and the code module running on
the underlay interface, for each of two adjacent layer interfaces.
More formally, for each layer interface $L_{low}$, we introduce a device driver module
$M$, and a new overlay interface $L_{high}$ abstracting the behaviors of $M$, and
show that $\llbracket{}M\rrbracket{}L_{low}$, is
a {\em contextual refinement} of the overlay interface $L_{high}$.
By repeating this strategy, we have developed a stack of layer interfaces,
with the top-most layer interface containing the full abstract behaviors of
the verified device drivers. Next, we introduce these layer interfaces and
the verified driver modules one by one. The implementations of various
concepts in Coq is shown in Fig.~\ref{fig:device:model}, with the concrete
examples used in the rest of sections listed on the right hand side.
We explain the concepts and implementations in more detail as we get to the
corresponding layer implementation.

\ignore{
A device layer stack is similar to the kernel layer stack (described in section \ref{sec:preli}) with the following
extensions: 1) Additional abstract states at the machine level (most bottom
layer) which corresponds to the device states. 2) Local logs to record the past
external events. 3) Raw primitives (\textsf{read} and \textsf{write}) for
modeling device access operational semantics.
}

\begin{figure}
	\[
	{
	\begin{tabu}[t]{l rll l}
%		{\centering \text{Type}}	&	\multicolumn{3}{c }{\centering \text{Definition}}	&	{\centering \text{Coq Presentation}} \\
%s		\hline	
		\textit{Event}	&	E	&	\in	&	\textsf{Recv}~(s:\textsf{list}~\textsf{char}) | \cdots	&	\texttt{Recv str / SendingCompAck / } \cdots \\
		\textit{Log}		&	\ell&	\in	&	\textsf{list}~\textit{Event}	&	\texttt{s.ltx / s.lrx / ...} \\
		\textit{EnvLog}	&	\ell^{\textsf{env}}&:& \text{All possible list of events.} & \text{Build into the `\textsf{next}' function.}\\
		\textit{ident}	&	id	&	\in	&	\textit{Nat} &  \texttt{Let serial\_init := 5950.}\\
		\textit{val}		&	v	&	:=	&	\textit{Vint}~v | ... | \textit{Vundef} & \texttt{Vint v / Vundef / ...}\\
		\textit{Abs}		&	s	&	\in	&	\textit{Type} \times L^{k}	&	\texttt{serial / ioapic / ...} \\
		\textit{Primitive}&	p	&	\in	&	\textsf{list}~\textit{val} \shortrightarrow \textit{Abs} \shortrightarrow L \shortrightarrow \textit{val} \shortrightarrow \textit{Abs} \shortrightarrow \textit{Prop}	&	\texttt{Function serial\_putc\_spec ...}	\\
		\textit{Invariant}&	\textsf{INV}	&	\in	&	\textit{Abs} \rightarrow \textsf{Prop} & \texttt{valid\_cons\_buf\_length: ...}\\
		\textit{Impl}&	\kappa	&:&	\multicolumn{2}{l }{\text{LAsm instruction list or ClightX function.}}  \\
		\textit{LayerIf}&	L	&	\in	& (ident \shortrightarrow Primitive)^{*}	& 
		\begin{tabu}{l}
		\texttt{Definition dserial :=}\\
		\quad \texttt{serial\_init} \mapsto \texttt{serial\_init\_spec}\\
		\quad \oplus~ ...
		\end{tabu}\\
		\textit{Module}&	M	&	\in	& (ident \shortrightarrow Impl)^{*}	& 
		\begin{tabu}{l}
		\texttt{Definition DSerial\_module :=}\\
		\quad \texttt{serial\_init} \mapsto \texttt{f\_serial\_init}\\
		\quad :: ~ ...
		\end{tabu}
	\end{tabu}
	}
	\]
	\vspace{-10pt}
	\caption{The formal model and Coq presentation of device abstract states and layer interfaces}
	\label{fig:device:model}
\end{figure}

\begin{table}
	\begin{tabular}[t]{p{.3\textwidth}p{.6\textwidth}}
	\hline
		{\centering Statement}	&	{\centering Description} \\
	\hline
		\texttt{x: T}					&	The type of \texttt{x} is \texttt{T}.\\
		\texttt{A -> B}					& 	Arrow type. Non-dependent product.\\
		\texttt{option \_}				&	Option type, either `Some v' or `None'.\\
		\texttt{hd :: tl}				&	List construction. Create a new list with an element `hd' and an old list `tl'.\\
		\texttt{l1 ++ l2}				& 	List concatenation. Create a new list with \texttt{l1} followed by \texttt{l2}.\\
		\texttt{s \{f: v\}}				&	Field update. Replace the value of field `f' in `s' with `v'. \\
		\texttt{Definition x: X := t.}	&	Constant definition. \\
		\parbox[t]{.3\textwidth}{\texttt{Function x ($a_1$:$T_1$)($a_2$:$T_2$)}\\
		\texttt{\hphantom{Func} ... : (rv: T) := $t$.}}
		 & Function definition with arguments $a_1, a_2, \cdots$, return value $rv$ and function body $t$.\\
		 \parbox[t]{.3\textwidth}{\texttt{Fixpoint x ($a_1$:$T_1$)($a_2$:$T_2$)}\\
		 \texttt{\hphantom{Fixp} ... : (rv: T) := $t$.}}
		 & As for function definition but $t$ can make recursive calls to x. \\
		\texttt{match t with} $p_1$ => $t_1$ | ...	\texttt{end}	&	Pattern matching, select $t_1$ if t matches with $p_1$.\\
		\texttt{if b then t else u}		&	Binary selection, b can be either \texttt{true} or \texttt{false}.\\
		\texttt{let x := $t$ in $u$}	&	Local binding.\\
		\hline
	\end{tabular}
	\caption{Common Coq terms, keywords and definitions}
	\label{table:coq}
\end{table}


Traditionally, refinement is proven by an upward simulation from implementation
to specification to ensure soundness, i.e., any property we prove on the overlay
specification is guaranteed to hold at underlay implementation. In our framework,
we also prove the upward simulation. But as a proof technique, we first prove the
downward simulation and later turn it into an upward one using the fact that our
machine semantics is deterministic with respect to the external events. We model
all non-deterministic behaviors by encapsulating all potential non-deterministic aspects
into event logs. And we prove our strong contextual refinement property with respect
to all possible combinations of event logs. This is how we guarantee
that our proof holds for all possible scenarios in the non-deterministic
executions. 

On the other hand, if the number of simulating underlay steps is not constrained, a
downward simulation could potentially be fulfilled by bogus implementations,  
We always insist writing the deep specifications for the overlay interface to
capture everything contextually observable for the underlay implementation.
Thus, the form of our high level specifications at overlay is not some random
high level relations vaguely restricting our underlay behaviors, but actual
precise specifications of what should be the exact outcome after running
corresponding abstract primitive, explained in terms of abstract states.
Thus, under same preconditions enforced by the primitive specification, the
behavior of overlay primitive and underlay implementation should be exactly
the same over the same external events. 

\subsection{The Lowest Level Layer Interface: MBoot}

The interface MBoot is the bottom-most layer interface used to model the
behaviors of the CPU and devices. In addition to the abstract states and
primitives in the original verified mCertiKOS, we have extended MBoot with the
new abstract states and read/write primitives for the serial port, I/O APIC, and
Local APIC devices as described in Sec.~\ref{sec:case_study}.

This layer interface introduces a set of layer invariants to enforce the
validity of various hardware states, and all the read/write primitives of the
devices are required to preserve the invariants. In the rest of the invariant
definitions, $\textsf{nth}$ is a list operation defined in Coq, which takes a
natural number $n$, a list $l$, a default value $v$, and returns the $n$'th
value in the list $l$. In the case when the index is invalid, it returns the
default value $v$. $\textsf{nth\_error}$ is a similar operation but returns
either a value or $\textsf{none}$ in the case of invalid index, instead of
returning the default value. $\textsf{Zlength}$ is another list operation that
returns the length of provided list as an integer.

\begin{invariant}[valid serial port]
The base address of the I/O port to access \textsf{COM1} is always
\texttt{0x3F8} (1016 in decimal) and cannot be changed by any primitive.
\begin{align*}
serial.\textsf{Base} = 1016
\end{align*}
\noindent{}\textnormal{The primitives \textsf{serial\_read} and
	\textsf{serial\_write} use $serial.\textsf{Base}$ to map the I/O address to the
	registers. Requiring this value to be constant ensures that there is no misuse
	of I/O addresses.}
\end{invariant}

\begin{invariant}[valid I/O APIC maxIntr]
The number of entries in the redirection table of an I/O APIC is less than 239.
\begin{align*}
0 \leq ioapic.\textsf{maxIntr} < 239
\end{align*}
\noindent{}\textnormal{The length of the redirection table in an I/O APIC varies
	depending on the hardware implementation. The actual value for a certain I/O
	APIC is static and stored in the I/O APIC Version Register[16:23], which
	corresponds to $ioapic.\textsf{maxIntr}$ in our device model. This invariant
	guarantees that the actual value of $ioapic.\textsf{maxIntr}$ is within a range.
%	It assures that the initialization of I/O APIC initializes
%	all the interrupt lines in the redirection table.
}
\end{invariant}

\begin{invariant}[valid I/O APIC masks length]
	The length of the mask array in an I/O APIC should be exactly $\textsf{maxIntr} + 1$.
\begin{align*}
\textsf{Zlength} (ioapic.\textsf{masks}) = ioapic.\textsf{maxIntr} + 1
\end{align*}
\noindent\textnormal{For the purpose of saving space, the hardware
	implementation of an I/O APIC makes each entry in the redirection table to be a
	64-bit register containing all the configuration items of a certain interrupt
	line, such as rtbl[i][0:7] for interrupt-vector, rtbl[i][8:10] for delivery
	mode, {\it etc}. In our abstract machine model, we extract each configuration
	item from all the entries and combine them into a list. This invariant requires
	the number of masks to be always equal to $\textsf{maxIntr} + 1$.}
\end{invariant}

\begin{invariant}[valid I/O APIC irq length]
	The length of the interrupt-vector in an I/O APIC should be exactly
$\textsf{maxIntr} + 1$.
\begin{align*}
\textsf{Zlength} (ioapic.\textsf{irqs}) = ioapic.\textsf{maxIntr} + 1
\end{align*}
\end{invariant}

\begin{invariant}[valid I/O APIC irq]
	The value of each interrupt-vector in an I/O APIC ranges from \texttt{0x10} to \texttt{0xFE}.
\begin{align*}
\forall n,~ 0 \le n \le ioapic.maxIntr \rightarrow 16 \leq \textsf{nth}~ n~ (ioapic.\textsf{irqs})~ 0 \leq 254
\end{align*}
\noindent\textnormal{The value in the interrupt-vector is the IRQ number raised
	on the connected CPU. This value can be configured during the runtime, and
	should be within the specified range.}
\end{invariant}

\begin{invariant}[valid interrupt states]
	If a device is not in the interrupt service mode (the execution is not in the
interrupt handler), the interrupt controller observed by the device should not
be in the interrupt service mode either. The state $in\_intr$ indicates whether
the system is currently handling an interrupt. Note that it will be abstracted
into corresponding logical states in the extended device objects during the
interrupt refinement process to enforce our isolation policy.
\[
\begin{array}{l}
\textsf{in\_intr} = \textsf{false} ~\rightarrow ~(lapic.\textsf{ISR} = \textsf{None} ~\wedge ioapic.\iota = None)
\end{array}
\]
\end{invariant}


\subsection{Layer Interface DConsoleBufferIntro}

On top of the underlay interface MBoot, we introduce and verify the circular
console buffer (for the serial driver) using the strategy illustrated in Chapter~
\ref{chapter:framework}. The layer interface DConsoleBufferIntro
corresponds to the intermediate layer interface $L_{mid}$ in Chapter~
\ref{chapter:framework}. At this layer, we also introduce the following
invariants on the intermediate console buffer (see Fig.~\ref{fig:spec:cons-buf})
at this layer interface. Recall that, by the design of a layer interface, the
invariants are required to hold at every moment of the system execution. Thus,
these invariants can be used as facts in any part of the verification.
Invariants are stronger notions than the preconditions in the specifications,
in the sense that the preconditions need to be validated during the primitive call while the
invariants are guaranteed to hold before and after calling the primitive.

\begin{invariant}[valid console buffer positions]
\begin{align*}
0 \leq d.rpos < \textsf{CB\_SIZE}
\wedge 0 \leq d.wpos < \textsf{CB\_SIZE}
\end{align*}
\end{invariant}

As explained in Chapter~\ref{chapter:framework}, the contextual
refinement from MBoot to DConsoleBufferIntro is achieved relatively easily as
the abstraction in the current layer interface is extremely similar to the
actual implementation. The concrete methodology is illustrated in Sec.
\ref{sec:dserialintro}, with a simpler example.

\subsection{Layer Interface DAbsConsoleBufferIntro}

This layer interface provides the higher level of abstraction of the console
buffer as a Coq list and corresponds to the layer interface $L_{high}$ in
Chapter~\ref{chapter:framework}. Detailed definitions and the refinement
proofs are omitted since they are already presented in Chapter~
\ref{chapter:framework}. This layer interface introduces a new layer
invariant on the abstract console buffer.

\begin{invariant}[valid console buffer length]
\begin{align*}
0 \leq \textsf{Zlength} (serial.cons\_buf) \leq \textsf{CB\_SIZE}
\end{align*}
\end{invariant}

\subsection{Layer Interface DSerialIntro}
\label{sec:dserialintro}

The next three layer interfaces are designated to the verification of serial
driver. On top of the underlay interface DAbsConsoleBufferIntro, we introduce a
global variable $\textsf{serial\_exist}$ of type $\textsf{bool}$ to indicate
whether the serial device is initialized or not, and provide a getter and setter
function called \textsf{get\_serial\_exist}, \textsf{set\_serial\_exist},
respectively. Accordingly, in \textsf{DSerialIntro}, we introduce a new abstract
state $\textsf{serial\_exist}: \textsf{bool}$ and two new getter and setter
primitives under the same names.

\ignore{
The layer interface DSerialIntro introduces following simple invariant.

\begin{invariant}[valid serial\_exist]
\begin{align*}
serial.serial\_exist = \textsf{true} ~\vee~ serial.serial\_exist = \textsf{false}
\end{align*}
\end{invariant}
}


\begin{figure}
	\lstinputlisting [language=C, multicols=2] {src/serial_exist.v}
	\caption{The specification and implementation of C function \texttt{set\_serial\_exist}}
	\label{fig:serial-exist}
\end{figure}

As an example, the source code (in C) and the specification (in Coq) of
\textsf{set\_serial\_exist()} are shown in Fig.~\ref{fig:serial-exist}.
Here, \textsf{Abs} is the type of abstract state in this layer interface,
\textsf{v} is the argument of the primitive, and the boolean function
$\textsf{zlt\_le}~a~b~c$ returns true if and only if $a< b\le c$, as
its name suggests. Furthermore, $d\{attr:val\}$ is our own notation
defined in the Coq, which indicates a new abstract states derived
from the original abstract states $d$, where the attribute $attr$ is
replaced by the new value $val$, but otherwise the same as $d$.
Note that the return type of the
specification function is an option type, and the specification gets stuck
(returns \textsf{None}) when the argument is not within the bound of 32-bit
integer. This is the requirement that any future caller of this primitive needs
to satisfy.

\begin{figure}
	\lstinputlisting [language=Caml] {src/serial_exist_lowspec.v}
	\caption{The low level specification of C function \texttt{set\_serial\_exist}}
	\label{fig:serial-exist-lowspec}
\end{figure}
 
The function \textsf{set\_serial\_exist} runs on a machine with the
underlay interface, and our goal is to prove the contextual refinement between
the code running on the underlay and the abstract overlay interface
DSerialIntro. The contextual refinement is proven in three steps.

First, we write a separate low level specification for the C code
\textsf{set\_serial\_exist} with the low level memory load and store operations
on the underlay interface, as shown in Fig.~\ref{fig:serial-exist-lowspec}.
Here, ``Inductive'' is the Coq keyword used to define a logical predicate
inductively, with the vertical bar ``$|$'' used to separate different cases
that can cause the relation to hold.
Then the inductively defined predicate
$\textsf{set\_serial\_exist\_low\_level\_spec}~ ge~ args~ m~ rval~ m'$
indicates that under the global environment ($ge$) mapping global variable
identifiers to their locations in the (CompCert) memory, given the argument list
$args$, the function changes the memory from $m$ to $m'$ with the return value
$rval$ ($\textsf{Vundef}$ if no return value). $\textsf{Mem.store}$ is an
operation in the CompCert memory model;  it takes the memory writing type, initial
memory to write to, the memory block, block offset, and a value, and returns the
new memory after writing the value on the location represented by the memory
block and offset on the initial memory. As shown in the figure, the
specification has two cases as in the source code, and in fact, it is very close
to the source code. Thus, it is relatively easy to show that the source code
satisfies the low level specification shown in
Fig.~\ref{fig:serial-exist-lowspec} and the proof can be achieved nearly
automatically by our proof tactics.

Second, we prove a simulation from the low level specification at underlay shown
in Fig.~\ref{fig:serial-exist-lowspec} and the overlay specification shown in
Fig.~\ref{fig:serial-exist}, with a refinement relation that trivially maps the
value of the global variable in the memory at underlay to the new abstract state
at overlay. This part of proof, which we call {\em data abstraction}, can
vary depending on the kind of concrete data structures in memory and the
abstracted states in the overlay interface. In the case of
$\textsf{set\_serial\_exist}$, the proof is very simple given that it is just a
single variable, but in other cases, the proof could be quite complex and we
may need to further split the proof into multiple layers as shown in the example
of the circular console buffer. The benefit is that after the data abstraction,
the modules and layer interfaces built on top of that could benefit greatly from
the simplicity in the abstracted form.

Last, we apply the correctness theorem of our CompCertX compiler to compile the
C source code and its simulation proof from the first step into appropriate form
in LAsm and link it with the simulation proof obtained in the second step, to
derive the desired final contextual refinement theorem. 

\subsection{Layer Interface DSerial}

On top of the underlay layer interface DSerialIntro, we introduce a new layer
interface DSerial to abstract three driver functions for the serial device. They
are \textsf{serial\_init} that initializes the serial device,
\textsf{serial\_getc} and \textsf{serial\_putc} that reads a character from and
writes a character to the serial device, respectively. The implementation of
\textsf{serial\_getc} and \textsf{serial\_putc} were shown in
Fig.~\ref{fig:serial_putc}, and the C source code for \textsf{serial\_init} is
shown in Fig.~\ref{fig:serial_init_c}.

\begin{figure}
	\lstinputlisting [language=C] {src/serial_init.c}
	\caption{The implementation of \texttt{serial\_init} in C}
	\label{fig:serial_init_c}
\end{figure}

\begin{figure}
	\lstinputlisting [language=Caml] {src/serial_putc.v}
	\caption{The specification of \texttt{serial\_putc} in Coq}
	\label{fig:serial_putc_v}
\end{figure}

A new layer invariant is introduced in DSerial to protect the configuration data.

\begin{invariant}[valid serial state]
	After initialization, the control registers in the serial device should
be properly configured.
\[
\begin{array}{ll}
	serial.\textsf{serial\_exist} = \textsf{true}  & ~\rightarrow \\
	& serial.\textsf{Baudrate} = 115200 ~\wedge~
	  serial.\textsf{Databits} = 8 ~\wedge~ \\
	& serial.\textsf{Stopbits} = 1 ~\wedge~ 
	  serial.\textsf{Parity} = \textsf{NoParity} ~\wedge~ \\
	& serial.\textsf{RxIntEnable} = \textsf{true} ~\wedge~ 
	  serial.\textsf{FIFO} = 1 ~\wedge~ \\
	&  serial.\textsf{Modem} = \textsf{DTR} + \textsf{RTS} + \textsf{OUT2}
\end{array}
\]

\end{invariant}

Both \textsf{serial\_getc} and \textsf{serial\_init} do not contain any loops
and are relatively easy to verify. Note that \textsf{serial\_init} detects the
existence of serial device, sets the proper configuration based on our hardware
connection parameters, and uses \textsf{set\_serial\_exist} to set the abstract
state \textsf{serial\_exist} when the initialization succeeds.

In the rest of this subsection, we present details on the function
\textsf{serial\_putc}. As shown in Fig.~\ref{fig:serial_putc}, the
implementation of \textsf{serial\_putc} involves a loop to poll the status of
TxBuf (through the \textsf{serial\_read} primitive of serial device) until it is
empty or reaches the maximum waiting steps. We assume the correctness of
hardware. Thus, we have a separate assumption on the event log stating the TxBuf
always become empty before the maximum iteration is reached.

The specification of the abstract \textsf{serial\_putc} primitive at the new
DSerial layer interface puts the sending character into the serial device's
TxBuf and updates the local log accordingly, as shown in
Fig.~\ref{fig:serial_putc_v}. It first performs some validation checks, including
whether the current execution status is in the critical mode, whether the serial
device is initialized (line 4) and whether the device is properly configured
(line 7). Then, it cases on whether the transmission buffer (TxBuf) is empty
(line 8). Here, we use the notation $d.attr$ to indicate the attribute
value with name $attr$ in the abstract state $d$, and $Z.eq\_dec$ is the
boolean function for integer equality.

Empty buffer indicates that the device is ready to send more characters, and in
this case, the primitive writes the proper contents to the abstract hardware
buffer. Our serial communication protocol requires a new line
(`\texttt{\textbackslash n}') to be a sequence of \textsf{LF} and \textsf{CR}, to
be compatible with most serial applications. The `if' expression at line 10 and
14 serves for this purpose. The local log $\ell_{tx}$  only records the
transmitting events, which separates from the receiving log $\ell_{rx}$. Since we
only consumed a single \textsf{SendingCompAck} event in this case, we simply
fetch the next transmitting event from the environment log ($\ell^{env}$) and
update the transmitting log. Recall that function $\textsf{next}$ is defined in
the Sec.~\ref{sec:model}. It takes a local log, finds the first relevant event
with respect to this log and returns a pair of event and the updated log.

When the buffer is non-empty, the device is still in the middle of transmitting
messages and the execution needs to wait until the transmission is complete. In
the implementation, there is a loop to poll the status of the transmitter, but
the specification simply writes the TxBuf with the sending message and updates
the log $\ell_{tx}$ to find the first \textsf{SendingCompAck} event in
$\ell^{env}$. This is achieved through the \textsf{next\_sendcomplete} fix-point
function shown in Fig.~\ref{fig:next_complete}. Note that
$\textsf{nextk}~\ell~k$ is a function to call $\textsf{next}$ $k$ times and only
returns the updated log.

\begin{figure}
	\lstinputlisting [language=Caml, multicols=2] {src/next_complete.v}
	\caption{The specification of \texttt{next\_sendcomplete} in Coq}
	\label{fig:next_complete}
\end{figure}

Next, we need to prove the contextual refinement between the specification and
the C code. To do that, we need to design a loop invariant that allows us to
prove that once the loop terminates, the transmission buffer (TxBuf) is empty so
that sending new messages to the serial will not overwrite the previous
messages. Recall that because of our assumption on the correctness of the
hardware, there is always an index $t$ between $0$ and $12800$, at which
iteration the loop terminates and serial device gets ready for the next write.
Thus, we have designed the following loop invariant:
\[
\begin{array}{l}
(0 \le i \le t ~\wedge~ serial_i = serial_0[\ell_{tx} \leftarrow \textsf{nextk}~ (serial_0.\ell_{tx},~~ i)] ~\wedge~ \textsf{lsr} = 0) ~\vee~ \\
(i = t + 1 ~\wedge~ serial_i = serial_0[\ell_{tx} \leftarrow \textsf{nextk}~ (serial_0.\ell_{tx}, ~~t + 1)][\textsf{TxBuf} \leftarrow nil] ~\wedge~ \textsf{lsr} = 1).
\end{array}
\]

Here, $serial_i$ indicates the value of abstract state $serial$ after the
$i^{th}$ iteration of the loop, where $serial_0$ indicates the initial state
before entering the loop.
With this loop invariant, the proof can be achieved with the help from
our automation tactic libraries.




\subsection{Layer Interface DConsole}

\begin{figure}
	\lstinputlisting [language=C, multicols=2] {src/cons_intr.c}
	\caption{The implementation of \texttt{serial\_intr\_handler} and \texttt{cons\_init} in C}
	\label{fig:console_c}
\end{figure}

\begin{figure}
	\lstinputlisting [language=Caml] {src/cons_intr.v}
	\caption{The specification of \texttt{serial\_intr\_handler} in Coq}
	\label{fig:cons_intr_v}
\end{figure}

The layer interface DConsole introduces two new primitives. One is
\textsf{cons\_init} which initializes the serial and the console buffer. The
other is the interrupt handler of the serial device
\textsf{serial\_intr\_handler}. Their C implementations are shown in
Fig.~\ref{fig:console_c}.

Once an interrupt signal is produced by the serial device indicating we have
newly received characters in the receiving buffer (RxBuf), the interrupt handler
repeatedly reads the characters one by one from RxBuf and writes them to the
console buffer until either all characters are received or the console buffer is
full of the newly received messages. It is achieved by repeatedly calling the
\textsf{serial\_getc} primitive (introduced at the DSerial) within a loop.

The specification of the \textsf{serial\_intr\_handler} primitive is shown in
Fig.~\ref{fig:cons_intr_v}. In contrast to the C implementation in
Fig.~\ref{fig:console_c}, the specification in Fig.~\ref{fig:cons_intr_v} does
not involve any fix points, but simply concatenates the whole receiving buffer
(RxBuf) into the abstract console buffer list, skipping extra characters in the
head of the list if necessary. This discrepancy between the implementation and
the clean loop-free specification introduces extra complexity in proving the
simulation between those two.

To prove the simulation between them, we first introduce a specification for the
loop and prove the loop body refines this specification. This is used later to
prove the whole function body containing the loop. The specification for the
loop body is a predicate on the abstract state $serial$ and the local
environment for storing the temporary variables in Clight, before ($serial, le$)
and after ($serial', le'$) the loop:
\[
\begin{array}{l}
	serial.\textsf{serial\_exist} = \textsf{true} \rightarrow \\
	serial.\textsf{RxBuf} = str \rightarrow \\
	\textsf{Forall}~~ \textsf{isChar}~~ str \rightarrow \\
	serial.(\textsf{cons\_buf}) = \mathsf{cons\_buf\_mid}~ str~ serial.(\textsf{cons\_buf})~ 0 \rightarrow \\
	le[\textsf{hasMore}] = \textsf{true} \rightarrow \\
	le[\textsf{t}] = 1 \rightarrow \\
	(serial'.lrx = \textsf{nextk} ~(serial.lrx, ~\textsf{Zlength} (tl~ str) * 2 + 1) \wedge \\
	 ~serial'.\textsf{cons\_buf} = \textsf{cons\_buf\_mid}~ str~ serial.\textsf{\textsf{cons\_buf}}~ (\textsf{Zlength}~ str)),
\end{array}
\]

\noindent{}where, $tl$ is the standard list operation that retrieves the tail of
a list, \textsf{Forall  isChar} is an inductive predicate to describe the
property that all the characters in $str$ is a valid character ($\textsf{fun}~x
\rightarrow 0 \le x \le 255$), and $\textsf{cons\_buf\_mid}~ str~ cb~ n$ is a
function shown in Fig.~\ref{fig:cons_buf_mid_v}. This function calculates the
console buffer when the first $n+1$ characters in $str$ is moved to $cb$.

\begin{figure}
	\lstinputlisting [language=Caml] {src/cons_buf_mid.v}
	\caption{The specification of \texttt{cons\_buf\_mid} in Coq}
	\label{fig:cons_buf_mid_v}
\end{figure}

The loop invariant is constructed as:

\[
\begin{array}{l}
	(~~0 \le i \le \textsf{Zlength} (tl~ str)~\wedge~
	le[\textsf{hasMore}] = 1 ~\wedge \\
	~~~serial_{i}.lrx = \textsf{nextk}~ (serial.lrx, ~~i \times 2) ~\wedge 
	serial_{i}.\textsf{RxBuf} = \textsf{skipn}~(i + 1)~str ~\wedge\\
	~~~serial_{i}.\textsf{cons\_buf} = \textsf{cons\_buf\_mid} ~~str ~~serial.\textsf{cons\_buf} ~~i~~) \\
	\vee \\
	(~~~i = \textsf{Zlength}~ str ~\wedge
	~le[\textsf{hasMore}] = 0 ~\wedge \\
	~~~serial_{i}.lrx = \textsf{nextk} ~(serial.lrx, ~~(\textsf{Zlength}~ str) \times 2 + 1) ~\wedge
	serial_{i}.\textsf{RxBuf} = nil ~\wedge \\
	~~~serial_{i}.\textsf{cons\_buf} = \textsf{cons\_buf\_mid}~~ str~~ serial_{i}.\textsf{cons\_buf}~~ (\textsf{Zlength}~ str)~~)
\end{array}
\]

This loop invariant has two parts: the first part shows the states during the
loop; and the second part shows the states when the loop terminates.

\subsection{Layer Interface DIOApic}

The layer interface DIOApic is built on top of I/O APIC machine interface. Four
primitives are introduced in DIOApic: \textsf{ioapic\_init},
\textsf{ioapic\_enable}, \textsf{ioapic\_mask}, and \textsf{ioapic\_unmask}.

The function \textsf{ioapic\_init} initializes the I/O APIC device.
Fig.~\ref{fig:ioapic_init} shows the C implementation of this primitive. It
includes a loop of calling \texttt{ioapic\_write()} to set the interrupt vectors
and masks. The specification of this primitive is shown in
Fig.~\ref{fig:ioapic_init_v}, which contains a fix-point to set related states.
In the figure, the abstract state \textsf{init} indicates whether the device has
been properly initialized. Every primitive other than the initialization
primitive has the precondition on the \textsf{init} being \textsf{true}.

To prove the simulation between these two loops, we first prove the refinement
between the loop body and the abstract \texttt{disable\_irq}. Then we prove the
simulation from the loop to the fix-point \textsf{ioapic\_init\_aux} with
following loop invariant:
\[
	0 \le i \le ioapic_{0}.\mathsf{maxIntr} \wedge
	ioapic_{i} = \mathsf{ioapic\_init\_aux} ~~ioapic_{0} ~~(Z.\textsf{to\_nat}~ i)
\]

\noindent{}From this loop invariant, we can show that when the loop
completes, the final state is consistent with the one in the specification.

\begin{figure}
	\lstinputlisting [language=Caml, multicols=2] {src/ioapic_init.v}
	\caption{The specification of \texttt{ioapic\_init} in Coq}
	\label{fig:ioapic_init_v}
\end{figure}

\ignore{
The specification of \textsf{ioapic\_init} carries a fixpoint which makes the
following verification to be more complex. In mCertiKOS, the kernel execution
consists of two stages: the booting stage and the running stage. \hao{please
	give them a good name.} In the booting stage, all the initialization functions
is invoked, and afterwards, the state $init$ in all the machine will turn from
\textsf{false} to \textsf{true}. 
}

We also introduce a new invariant in this layer interface.

\begin{invariant}[valid I/O APIC state]
	After initialization, the value of interrupt vector corresponding to interrupt line $n$ should be equal to $n + \textsf{IRQ0}$ and this interrupt line should be either masked or unmasked.
	\[
	\begin{array}{l}
	init = \textsf{true} \rightarrow \\
	\forall n \in \mathbb{N}, ~ v \in \mathbb{Z},~
	\textsf{nth\_error}~ (ioapic.\textsf{irqs})~ n = value~ v \rightarrow \\
	( v = Z.of\_nat~ n + \textsf{IRQ0}  ~\wedge~ 
	\exists~ b \in \mathbb{B},~ \textsf{nth\_error} (ioapic.\textsf{masks})~ n = value~ b)
	\end{array}
	\]
	\noindent\textnormal{In the mCertiKOS kernel, we allocate sequential entries from
		the interrupt descriptor table (IDT) for the IRQs. The range is from
		\textsf{IRQ0} to $\textsf{IRQ0} + ioapic.\textsf{maxIntr}$. \textsf{IRQ0} is the
		first IRQ number which is also known as the global system interrupt (GSI). The
		initialization sets the correct values of interrupt vectors which match the IDT
		entries so that the correct interrupt handler can be called when an IRQ occurs.
		The mapping from interrupt lines to IRQ numbers is static in mCertiKOS, so this
		invariant protects this mapping on the I/O APIC. }
\end{invariant}

Primitive \textsf{ioapic\_enable} is used to set the routing configuration for
an interrupt line in order to make it ready for serving the incoming interrupts.
The implementation and specification of \textsf{ioapic\_enable} are shown in
Fig.~\ref{fig:ioapic_enable_v}. Because mCertiKOS only uses one mode of the
interrupt delivery, we only check the validity of given parameters of $lapicid$,
$trigger$, and $polarity$, but do not model the functionality of these
parameters in our device. The \textsf{ioapic\_write} primitive also requires
these parameters to contain exactly the same values.

The primitives \textsf{ioapic\_mask} and \textsf{ioapic\_unmask} are used to
mask and unmask a particular interrupt line designated for a device. It is not
allowed to designate a single interrupt line for multiple devices. Thus, the
masking status of each interrupt line could be later abstracted into the
internal state of corresponding abstract device object to enforce our isolation
policy. The verification of \textsf{ioapic\_mask} and \textsf{ioapic\_unmask}
is similar to that of \textsf{ioapic\_enable}.

The verification of the four functions introduced in DIOApic is reasonably
simple, and can be automated using our tactics.

\begin{figure}
	\lstinputlisting [language=c, multicols=2] {src/ioapic_enable.c}
	\caption{The implementation (in C) and specification (in Coq) of \texttt{ioapic\_enable}}
	\label{fig:ioapic_enable_v}
\end{figure}

\subsection{Layer Interface DLApic}

Two primitives are introduced under the layer interface DLApic:
\textsf{lapic\_init} and \textsf{lapic\_eoi}. The verification of
\textsf{lapic\_init} is similar to \textsf{ioapic\_init}. Recall that during the
later contextual refinement of interrupt models, the operations of two interrupt
controllers will be merged and canceled to derive the abstract interrupt model,
where an interrupt triggered by the serial device is only related to the serial
device object, and is independent from either of the interrupt controllers or
the CPU.


\subsection{Above DLApic}

Finally, we introduce a new layer interface to introduce the abstract interrupt
model (as shown in Sec. \ref{sec:interrupt}), and more layer interfaces
for the driver-related system calls.




