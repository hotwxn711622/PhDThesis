\label{chapt:concl}

In this thesis, I have presented a novel compositional framework for reasoning about
the end-to-end functional correctness of a certified concurrent
interruptible kernel with device drivers. 
The certified abstraction layers bring modular design into our proofs,
minimizing unnecessary boilerplate and isolation proofs, and maximizing
proof reuse.  The formalization of interrupts follows the
abstraction-layer-based approach and includes a realistic hardware
interrupt model and an abstract model of interrupts (which is suitable
for reasoning about interruptible code). We have proved that the two
interrupt models are contextually equivalent.  We have successfully
extended an existing verified non-interruptible kernel with our
framework and turned it into an interruptible kernel with verified
device drivers with minimal proof change.
The extended support on fine-grained concurrency allows us
to reason about each local thread with fine-grained locking separately,
and formally link the local proofs to derive the global properties of
the entire system. 
The framework allows us to perform the implementation, specification, and proofs
all in a unified framework (realized in the Coq proof assistant), yet
the mechanized proofs verify the correctness of the assembly code that
can run on the actual hardware. To the best of our knowledge, this is
the first framework with support on end-to-end verification of low level OS kernel
with device drivers, interrupts, and fine-grained concurrency, in a
unified framework.

