Formal verification is the only known way of building bug-free or hacker-proof
program. However, due to its prohibitive associated costs, formal verification
has been rarely considered as an option in building robust large-scale system software.
Practical system software normally consist of highly correlated interdependent subsystems,
with complex invariants that need to be globally maintained. To reason about correctness of
a piece of program, we not only need to show that the program in consideration
satisfies the invariants and the specification, but also prove that the invariants cannot be
accidentally broken by other parts of the system, e.g., via pointer manipulation.
Furthermore, we often have some snippet of code that temporarily breaks the invariants,
and re-establishes them later, which could explode the reasoning complexity of such code.
Even worse, many complex system software contains device drivers to work with the devices,
which brings a major challenge of handling device interrupts;
and consists of multiple threads running on multiple CPUs concurrently. 
This forces us to further reason about arbitrary interactions and interleaved executions among
different devices, interrupts, and programs running on different CPUs, which could quickly make the verification
task intractable.

In this dissertation, we present simple, yet extremely powerful automation engine for effectively
verifying complex system software. It is simple because it solely focuses on providing strong automation
support for verifying functional correctness properties of C source programs, while taking the
isolation and invariant properties as given, and separately provide systematic approach
for guaranteeing the isolation among different in-memory data, and proving invariants, completely
at the logical level. The engine also contains novel way of representing devices and drivers, and
simulation-based approach for turning the low level interrupt model into an equivalent abstract model which is
suitable for reasoning about interruptible code. Furthermore, the engine provides a new way of representing
concurrent shared states into sequence of external I/O events, and allows us to verify each main piece of
concurrent program as if they were sequential and provide separate logical framework to effectively
reason about interleaved executions. This very modular design allows us to be able to reason about each
aspect of the system separately, and while each of the reasoning task looks unbelievably simple, we could combine
all the proofs to obtain proofs of properties about complex system software.
An OS kernel is very typical example of complex low level system software with highly interdependent modules.
To illustrate effectiveness of our approach, using all the tools, we have developed fully verified feature-rich
operating system kernel with machine checkable proof in the Coq proof assistant.



\ignore{
An operating system (OS) kernel forms the lowest level of any system
software stack. The correctness of the OS kernel is the basis for the
correctness of the entire system. Recent efforts have demonstrated the
feasibility of building formally verified general-purpose kernels, but
the cost of such verification is still prohibitive. Furthermore,
it is unclear how to extend their work to verify the functional
correctness of device drivers, due to the non-local effects of
interrupts. Last not the least,  
complete formal verification of a non-trivial concurrent
OS kernel is widely considered a grand challenge, and
none of these systems have addressed
the issues of concurrency, despite the fact that majority
of OS kernels run on multicore machine nowadays.

This thesis presents a novel compositional framework
for building certified interruptible and concurrent OS kernels with device
drivers. We present formal study of abstraction layers, where
each abstraction layer specifies the precise functionality
of underlying implementation with clear assumptions about its
external context. The framework provides
systematic ways of verifying complex system software like OS kernel,
by specifying, programming, verifying, and composing the abstraction
layers. To support device drivers and interrupts,
we provide a general device model that can be instantiated
with various hardware devices, and a realistic formal model of
interrupts, which can be used to reason about interruptible code. 
For concurrency, we have a novel event based model for representing
shared states,
which allows us to verify each process locally by 
making proper assumptions over other processes, and later
merge all individual proofs together to obtain global properties
across the entire concurrent system.

To demonstrate the effectiveness of our new approach, we have
successfully developed a practical concurrent operating system
kernel in our framework.
The kernel consists of 6,500 lines of C and assembly code, runs on stock
x86 multicore machines, and doubles as a hypervisor and boots multiple instances
of Linux as guest on different CPUs. 
The implementation, modeling, specification, and
  proofs are all done in a unified framework (realized in the Coq
  proof assistant), yet the machine-checkable proofs verify the
  correctness of the assembly code that runs on the actual
  hardware.
To the best of our knowledge, this is the first framework with support
on end-to-end verification of low level OS kernel with device drivers,
interrupts, and fine-grained concurrency, in a unified framework.
}
