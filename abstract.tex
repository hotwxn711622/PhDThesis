An operating system (OS) kernel forms the lowest level of any system
software stack. The correctness of the OS kernel is the basis for the
correctness of the entire system. Recent efforts have demonstrated the
feasibility of building formally verified general-purpose kernels, but
the cost of such verification is still prohibitive. Furthermore,
it is unclear how to extend their work to verify the functional
correctness of device drivers, due to the non-local effects of
interrupts. Last not the least,  
complete formal verification of a non-trivial concurrent
OS kernel is widely considered a grand challenge, and
none of these systems have addressed
the issues of concurrency, despite the fact that majority
of OS kernels run on multicore machine nowadays.

This thesis presents a novel compositional framework
for building certified interruptible and concurrent OS kernels with device
drivers. We present formal study of abstraction layers, where
each abstraction layer specifies the precise functionality
of underlying implementation with clear assumptions about its
external context. The framework provides
systematic ways of verifying complex system software like OS kernel,
by specifying, programming, verifying, and composing the abstraction
layers. To support device drivers and interrupts,
we provide a general device model that can be instantiated
with various hardware devices, and a realistic formal model of
interrupts, which can be used to reason about interruptible code. 
For concurrency, we have a novel event based model for representing
shared states,
which allows us to verify each process locally by 
making proper assumptions over other processes, and later
merge all individual proofs together to obtain global properties
across the entire concurrent system.

To demonstrate the effectiveness of our new approach, we have
successfully developed a practical concurrent operating system
kernel in our framework.
The kernel consists of 6,500 lines of C and assembly code, runs on stock
x86 multicore machines, and doubles as a hypervisor and boots multiple instances
of Linux as guest on different CPUs. 
The implementation, modeling, specification, and
  proofs are all done in a unified framework (realized in the Coq
  proof assistant), yet the machine-checkable proofs verify the
  correctness of the assembly code that runs on the actual
  hardware.
To the best of our knowledge, this is the first framework with support
on end-to-end verification of low level OS kernel with device drivers,
interrupts, and fine-grained concurrency, in a unified framework.
