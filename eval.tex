\section{Evaluation and Lessons Learned}
\label{sec:lessons}

\paragraph{What We Have Proved}
The final theorem we proved for our kernel is the contextual refinement
relation between
our lowest level hardware machine model $\texttt{MBoot}$ (which defines
the x86 instructions, the serial device, and the I/O APIC and Local APIC
devices, etc.), and the top level machine $\texttt{mCertiKOS}$ (which
defines the abstract system call interface).  Let
$\sem{\rm{}x86}{\cdot}$ and $\sem{\rm{}mCertiKOS}{\cdot}$ denote the
whole-machine semantics of each machine model, and $K$ denote the
(assembly) source code of mCertiKOS, then the theorem is formalized
as:

\begin{theorem}
 $\forall P,\;\sem{\rm{}x86}{K\join{}P}\Refrel{}\sem{\rm{}mCertiKOS}{P}$.
\end{theorem}

The theorem states that for any kernel/user/guest/host
context program $P$, there is a simulation between program $P$ running
on top of the top level abstract machine $\texttt{mCertiKOS}$, and the
program $P$ linked with the mCertiKOS source code $K$, running
atop the bottom-most machine $\texttt{x86}$.

The abstraction layers also define the data invariants that are proved
to hold at any moment of the whole program execution. Some example
invariants are: the console's circular buffer is always well\-formed,
and the interrupt controller states are always consistent, {\it etc}.

Besides this, our framework automatically derives that all the system
calls always run safely and terminate; there are no code injection
attacks, no buffer overflows, no null pointer access, no integer
overflows, {\it etc}.

\paragraph{Isolation}
We take the existing implementation of the CertiKOS infrastructure
\cite{dscal15}, and extend it with our device and interrupt models.
On top of the extended machine model, we have verified a subset of the
device drivers in mCertiKOS with 10 abstraction layers. Some layers
are introduced to verify concrete driver implementation, while others
are introduced purely for logical abstraction (e.g., from a circular
console buffer implementation in memory to an abstract list, from the
hardware interrupt model to the abstract interrupt model enforcing
isolation, {\it etc}).  These abstraction layers are inserted into the
existing layers of mCertiKOS as a certified plugin.  Thanks to our
isolation policy, this does not invalidate most of the existing proofs
of mCertiKOS, and the integration only required minimal effort, despite
the existing mCertiKOS proofs being unaware of interrupts.


\paragraph{Execution Model and Completeness}
The majority of our device drivers are specified and verified at C
level, then compiled by our CompCertX compiler. 
The entire kernel (both C and assembly)
source code, together with the source code for the verified compiler,
are extracted into an OCaml program through Coq's extraction
mechanism. When this program gets executed, it compiles the extracted
C source code into the assembly, and merges it with the existing
assembly kernel source code, to produce a piece of assembly code
corresponding to our verified kernel.  Thus, our deliverable comes
with a piece of assembly code for the entire verified kernel, a high
level deep specification of various kernel behaviors, and a machine
checkable proof object stating the assembly code running on the actual
hardware satisfies the high level specification.

The verified assembly code is then linked with the rest of kernel code
(the boot loader and remaining unverified drivers) to produce the
actual binary image of the OS. The resulting kernel is practical: it
runs on stock x86 hardware and the hypervisor version with the Intel vt-x
support can successfully boot a unmodified version of Linux as guest.

\paragraph{Verification Effort} 
Using our general device interface, we have modeled a serial device
and two interrupt controller devices. On top of these device models,
we have verified the related drivers and interrupt handlers.  The
entire verification effort consists of roughly 20k lines of Coq code
added to the existing mCertiKOS verification code base.  Regarding the
specification, there are 510 lines of code used to specify the machine
model including the device hardware, and 126 lines of code for
specification of the additional system call interfaces. There are
additional 9,829 lines of Coq code that were used to define auxiliary
definitions, lemmas, theorems, invariants, {\it{}etc}. Note that these
9,829 lines of definitions are outside our TCB, thus does not need
to be trusted.  In terms of proof size, there are 3,671 lines of Coq
code for the layer refinement proofs, 3,589 lines for code
verification, 1,802 lines for proving invariants, and 307 lines for
linking different modules together.

The entire verification effort took roughly 7 person months, the
majority of which went into the design and development of the
framework itself, including the extended machine model, general device
framework, the interrupt refinement, and the tactic libraries for
automating most of the non-intellectual parts of verification task.
We anticipate the cost of verification for future drivers would be
dramatically reduced.

\paragraph{Bugs Found}
An extended version of the mCertiKOS kernel has been deployed in a
practical system that is used in the context of a large DARPA-funded
research project \cite{dscal15}.  Yet, through the verification of the
console driver, we found a critical bug which may lead to the loss of
many characters received from the serial device. The bug was in the
implementation of the circular console buffer, where, in some rare
cases, the read and write positions to the buffer array overlap,
causing the entire contents in the buffer to be lost. The bug was
caught when we tried to establish the contextual refinement between
the concrete implementation of the circular buffer and its abstract
list representation.

Another bug was found in the code for initializing the serial device,
where the interrupt was not configured correctly by accidentally
setting the Interrupt Enable Register (IER) before the DLAB was unset.
This was caught when we tried to prove the initialization code against
its specification.



